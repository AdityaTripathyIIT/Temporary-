
\let\negmedspace\undefined
\let\negthickspace\undefined
\documentclass[journal,12pt,twocolumn]{IEEEtran}
\usepackage{cite}
\usepackage{amsmath,amssymb,amsfonts,amsthm}
\usepackage{algorithmic}
\usepackage{graphicx}
\usepackage{textcomp}
\usepackage{xcolor}
\usepackage{txfonts}
\usepackage{listings}
\usepackage{enumitem}
\usepackage{mathtools}
\usepackage{gensymb}
\usepackage{comment}
\usepackage[breaklinks=true]{hyperref}
\usepackage{tkz-euclide} 
\usepackage{listings}
\usepackage{gvv}                                        
\usepackage[latin1]{inputenc}                                
\usepackage{color}                                            
\usepackage{array}                                            
\usepackage{longtable}                                       
\usepackage{calc}                                             
\usepackage{multirow}                                         
\usepackage{hhline}                                           
\usepackage{ifthen}                                           
\usepackage{lscape}
\usepackage{tabularx}
\usepackage{array}
\usepackage{float}


\newtheorem{theorem}{Theorem}[section]
\newtheorem{problem}{Problem}
\newtheorem{proposition}{Proposition}[section]
\newtheorem{lemma}{Lemma}[section]
\newtheorem{corollary}[theorem]{Corollary}
\newtheorem{example}{Example}[section]
\newtheorem{definition}[problem]{Definition}
\newcommand{\BEQA}{\begin{eqnarray}}
\newcommand{\EEQA}{\end{eqnarray}}
\newcommand{\define}{\stackrel{\triangle}{=}}
\theoremstyle{remark}
\newtheorem{rem}{Remark}
\usepackage{textcomp}
\usepackage{enumitem}
\usepackage[a4paper, portrait, margin = 1in]{geometry}
\title{Trignometric Functions and Equations}
\author{EE24BTECH11001- ADITYA TRIPATHY}
\usepackage{graphicx}
\usepackage{multicol}

\begin{document}
\maketitle

\section*{\textbf{A: Fill In The Blanks}}


\begin{enumerate}
	


	\item Suppose $\sin{x}^{3}xsin3x = \sum_{m=0}^{n} C_m cosx $ is an identity in $x$, where $C_0, C_1, \cdots , C_n$ are constants and $C_n \neq 0$ then the value of n is 
	\item Suppose $\sin^3{x}\sin3x = \sum_{m=0}^{n} C_m \cos x $ is an identity in $x$, where $C_0, C_1, \cdots , C_n$ are constants and $C_n \neq 0$ then the value of n is
		\begin{flushright}
			\brak{\textit{1981 - 2 Marks}}
		\end{flushright}
	


	\item The solution set of the system of equations \(x + y = \frac{2\pi}{3}\), \(\cos x + \cos y = \frac{3}{2}\), where $x$ and $y$ are real,is 
		\begin{flushright}
  			\brak{\textit{1987 - 2 Mark}}
  		\end{flushright}
  	\item The set of all $x$ in the interval $[0,\pi]$ for which $2 \sin^2 x -3\sin x +1 \ge 0$, is 
		\begin{flushright}
			\brak{\textit{1987 - 2 mark}}
		\end{flushright}
  
  
	
	\item The sides of a triangle in a given circle subtend angles $\alpha$, $\beta$, $\gamma$. The minimum value of arithmetic mean of $\cos \brak{\alpha + \frac{\pi}{2}}$, $\cos \brak{\beta + \frac{\pi}{2}}$, $\cos \brak{\gamma + \frac{\pi}{2}}$ is equal to 
		\begin{flushright}
			\brak{\textit{1987 - 2 Marks}}
		\end{flushright}
  
	

	\item The value of \[\sin\frac{\pi}{14}\sin\frac{3\pi}{14}\sin\frac{5\pi}{14}\sin\frac{7\pi}{14}\sin\frac{9\pi}{14}\sin\frac{11\pi}{14}\sin\frac{13\pi}{14}\] is equal to  
		\begin{flushright}
			\brak{\textit{1991 - 2 Marks}}
		\end{flushright}
  
  
	
	\item If $K = \sin(\frac{\pi}{18})\sin(\frac{5\pi}{18})\sin(\frac{7\pi}{18})$ then the numerical value of $K$ is  
		\begin{flushright}
			\brak{\textit{1993 - 2 Marks}}
		\end{flushright}
  
  
	
	\item If $A > 0, B>0$ and $A + B = \frac{\pi}{3}$, then the maximum value $\tan A \tan B$ is  
		\begin{flushright}
			\brak{\textit{1993 - 2 Marks}}
		\end{flushright}
  
	

	\item General value of $\theta$ satisfying the equation $\tan^{2}\theta +\sec2\theta = 1$ is 
		\begin{flushright}
			\brak{\textit{1996 - 1 Mark}}
		\end{flushright}
  
	

	\item The real roots of the equation $\cos^{7}x + \sin^{4}x = 1$ in the interval $\brak{-\pi,\pi}$ are 
		\begin{flushright}
			\brak{\textit{1997 - 2 Marks}}
		\end{flushright}



\end{enumerate}  



\section*{\textbf{B: True / False}}



\begin{enumerate}



	\item If $\tan A = \frac{1-\cos B}{\sin B}$ , then $\tan 2A = \tan B$ 
		\begin{flushright}
			\brak{\textit{1981 - 1 Marks}}
		\end{flushright}



	\item There exists a value of $\theta$ between $0$ and $2\pi$ that satisfies the equation $\sin^{4}\theta -2\sin^{2}\theta-1=0$. 
		\begin{flushright}
			\brak{\textit{1984 - 1 Marks}}
		\end{flushright}



\end{enumerate}



\section*{\textbf{C :MCQs w ith One Correct Answer}}


\begin{enumerate}



	\item If $\tan\theta =-\frac{4}{3}$ then $\sin \theta$ is 
		\begin{flushright}
			\brak{\textit{1979}}
		\end{flushright}
  
		\begin{enumerate}[label=(\alph*)]
			\begin{multicols}{2}
				\item $\frac{-4}{5}$ but not $\frac{4}{5}$ 
				\columnbreak
				\item $\frac{4}{5}$ or $\frac{-4}{5}$ 
			\end{multicols}
			\begin{multicols}{2}
				\item $\frac{4}{5}$ but not $\frac{-4}{5}$ 
	  			\columnbreak
				\item None of These 
			\end{multicols}
		\end{enumerate}
  

  
	\item If $\alpha+ \beta +\gamma = 2\pi$ 
		\begin{flushright}
			\brak{\textit{1979}}
		\end{flushright}
  
		\begin{enumerate}[label=(\alph*)]
  
  
			\item $\tan\frac{\alpha}{2} + \tan\frac{\beta}{2} + \tan\frac{\gamma}{2} = \tan\frac{\alpha}{2}\tan\frac{\beta}{2}\tan\frac{\gamma}{2}$
  
  
			\item $\tan\frac{\alpha}{2}\tan\frac{\beta}{2} + \tan\frac{\beta}{2}\tan\frac{\gamma}{2}+ \tan\frac{\gamma}{2}\tan\frac{\alpha}{2} = 1$
  
			\item $\tan\frac{\alpha}{2} + \tan\frac{\beta}{2} + \tan\frac{\gamma}{2} = -\tan\frac{\alpha}{2}\tan\frac{\beta}{2}\tan\frac{\gamma}{2}$
  
			\item None of These
  
 
		\end{enumerate}
  
  

	\item Given $A = \sin^{2}\theta + \cos^{4}\theta $ then for all real values of $\theta$ 
		\begin{flushright}
			\brak{\textit{1980}}
		\end{flushright}
  
		\begin{enumerate}[label=(\alph*)]
			\begin{multicols}{2}
				\item $1 \le A \le2$
	  			\columnbreak
  				\item $\frac{3}{4} \le A\le 1$ 
			\end{multicols}
		  	\begin{multicols}{2}
				\item $\frac{13}{16} \le A\le 1$
	  			\columnbreak
				\item $\frac{3}{4} \le A\le \frac{13}{16}$ 
	  		\end{multicols}
		\end{enumerate}
  


	\item The equation $2\cos^{2}\frac{x}{2}\sin^{2}x = x^{2} +x^{-2}$  
		\begin{flushright}
      			\brak{\textit{1980}}
		\end{flushright}
      
		\begin{enumerate}[label=(\alph*)]
			\begin{multicols}{2}
			\item no real solution
	  		\columnbreak
	  		\item one real solution
			\end{multicols}
			\begin{multicols}{2}
			\item more than one real solution 
	  		\columnbreak
			\item None of these
			\end{multicols}
		\end{enumerate}
  


	\item The general solution to the trignometric equation $ sinx + cosx =1$ is given by
		\begin{flushright}
			\brak{\textit{1981 - 2 Marks}}
		\end{flushright}
  
		\begin{enumerate}[label=(\alph*)]
			\item $x=2n\pi;n=0,\pm1,\pm2 \cdots$
			\item  $x = 2n\pi + \frac{\pi}{2}, n = 0, \pm 1, \pm 2 \cdots $
			\item $x=n\pi+\brak{-1}^{n}\frac{\pi}{4},n = 0,\pm 1,\pm 2 \cdots $ 
			\item none of these
		\end{enumerate}
  


	\item The value of the expression $\sqrt{3}\cosec 20^{\circ} - \sec 20^{\circ} $ is equal to 
		\begin{flushright}
      			\brak{\textit{1988 - 2 Marks}}
		\end{flushright}
  
			\begin{enumerate}[label=(\alph*)]
				\begin{multicols}{2}
				\item 2 
	  			\columnbreak
  				\item $2\sin 20^{\circ}/\sin40^{\circ}$
				\end{multicols}
		  		\begin{multicols}{2}
  				\item 4 
				\columnbreak
				\item $2\sin20^{\circ}/\sin40^{\circ}$
		  		\end{multicols}
			\end{enumerate}
  
  
\end{enumerate}
\end{document}

